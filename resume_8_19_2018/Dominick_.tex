%%%%%%%%%%%%%%%%%%%%%%%%%%%%%%%%%%%%%%%%%
% Medium Length Professional CV
% LaTeX Template
% Version 2.0 (8/5/13)
%
% This template has been downloaded from:
% http://www.LaTeXTemplates.com
%
% Original author:
% Trey Hunner (http://www.treyhunner.com/)
%H
% Important note:
% This template requires the resume.cls file to be in the same directory as the
% .tex file. The resume.cls file provides the resume style used for structuring the
% document.
%
%%%%%%%%%%%%%%%%%%%%%%%%%%%%%%%%%%%%%%%%%

%----------------------------------------------------------------------------------------
%	PACKAGES AND OTHER DOCUMENT CONFIGURATIONS
%----------------------------------------------------------------------------------------

\documentclass{resume} % Use the custom resume.cls style

\usepackage[left=0.75in,top=0.6in,right=0.75in,bottom=0.6in]{geometry} % Document margins

\name{Dominick Hambrick} % Your name
\address{685 Providence Main Street  , Apt 31 \\ Huntsville, Al 35806} % Your address
\address{(256)~$\cdot$~424~$\cdot$~2254 \\ dominick.hambrick@gtri.gatech.edu} % Your phone number and email

\begin{document}


%----------------------------------------------------------------------------------------
%	WORK EXPERIENCE SECTION
%----------------------------------------------------------------------------------------

\begin{rSection}{Experience}

\begin{rSubsection}{Georgia Tech Research Insititute}{Feb 2014 - Present}{Research Scientist I}{Huntsville , Al}
In my role as a Research Scientist , I have had the pleasure of contributing to a number of technical projects, ranging from
developing software platforms that process and visualize Lidar data, to porting legacy Fortran code to CUDA C. 

Currently I'm serving on a software team to rearchitect a networking test tool for the US Army's IBCS program. In this role, I
am responsible for developing and  testing a C++ distributed backend. I've also engaged in front end web work for visualizing
real-time data.

Some of previous my work includes:
\begin{itemize}
 \item Ported  performance critical sections of a legacy Fortran  simulation platform to CUDA C.
\item Implemented advanced mesh parametrization algorithms in C++ for the MDA.
\item Helped support refactor of a GUI for a realtime Radar Display for the Spanish Government.
\item Established a Software Defined Radio Testbedy for the Applied Systems Lab in anticipation of future costomer need.
\item As culmination of the effort of establing the SDR Testbed, I developed a prototype cognitive radio  platform using GNURadio and Python/Numpy.
\item Implemented the low level networking and memory interfaces in C++ for a real-time Lidar data processing and visualization platform. I also architected and implemented the multithreading scheme for this platform.
\end{itemize}
\end{rSubsection}

\begin{rSubsection}{Georgia Tech Research Insititute}{Jan  2012 - Feb 2014}{Research Engineer}{Huntsville, AL}

%As a Research Engineer at GTRI , I had the privalege of working on cutting edge work research in the field of
%radar multiple hypothesis tracking  algorithms in support of the Missile Defense Agency (MDA). Under direction of a world 
%renowned radar expert, Dr Dale Blair, I contributed to the development, implementation, and characterization of a Multiple Hypothesis
%tracking simulation platform in Matlab. This simulation platform allowed us to develop novel algorithms for sensor fusion and bias estimation.
%This work ultimately culminated in presenting our to at an IEEE conference, with a paper detailing our method and results subsquently published
\item in an IEEE Journal.



\end{rSubsection}

\end{rSection}

\begin{rSection}{Technical Strengths}

\begin{tabular}{ @{} >{\bfseries}l @{\hspace{6ex}} l }
Computer Languages & C, Cuda C,C++,Python , Javascript, Bash ,Matlab \\
Protocols \& APIs &  JSON, REST,YAML,Google Protocol Buffers ,NUMPY, GNU Radio \\
Databases & MongoDB \\
Tools & Git,Jira,SVN,CMake, Vim, Emacs \\
Frameworks & Boost , Boostrap , Node.js , gRPC , opengl  \\
\end{tabular}
\end{rSection}

%----------------------------------------------------------------------------------------
%	EDUCATION SECTION
%----------------------------------------------------------------------------------------
\newpage
\begin{rSection}{Education}

{\bf University of Alabama, Huntsville} \hfill {\em \bf Graduation Date: Dec 2010} \\ 
B.S. in Physics  \\
Minor in Mathematics \smallskip \\
Overall GPA: 3.1 \\
\\
{\bf Georgia Institute of Technology} \hfill {\em \bf Graduation Date: N/A} \\ 
M.S. in Computer Science  \\
\\
%\begin{itemize}{Relevant Coursework}
%\item CS 6476 Computer Vision
%\item CS 6210 Advanced Operating Systems
%\item CSE 6220 High Performance Computing 
%\end{itemize}

\end{rSection}
%----------------------------------------------------------------------------------------
%	TECHNICAL STRENGTHS SECTION
%----------------------------------------------------------------------------------------





%----------------------------------------------------------------------------------------
%	EXAMPLE SECTION
%----------------------------------------------------------------------------------------

%\begin{rSection}{Section Name}

%Section content\ldots

%\end{rSection}

%----------------------------------------------------------------------------------------

\end{document}
